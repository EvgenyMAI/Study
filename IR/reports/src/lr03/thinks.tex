\section{Выводы}

В ходе выполнения лабораторной работы я получил практический опыт создания высокопроизводительных компонентов для обработки текста.

\begin{enumerate}
	\item \textbf{Интеграция языков:} я научился связывать Python и C++ через механизм \texttt{ctypes}, что позволяет объединить скорость компилируемого языка с удобством разработки на скриптовом языке.
	\item \textbf{Работа с Unicode:} я разобрался в устройстве кодировки UTF-8 и алгоритмах её посимвольного разбора и понял важность корректной обработки многобайтовых символов (особенно кириллицы) при реализации строковых алгоритмов вручную.
	\item \textbf{Анализ данных:} я научился оценивать качество токенизации не только визуально, но и статистически; анализ частотного словаря и длины токенов позволил выявить недостатки эвристик и наметить пути их исправления.
	\item \textbf{Профилирование:} я убедился, что \enquote{узким местом} в таких системах часто становится не сам алгоритм, а операции ввода-вывода (чтение из базы данных), что показало важность комплексной оптимизации производительности.
\end{enumerate}

\pagebreak
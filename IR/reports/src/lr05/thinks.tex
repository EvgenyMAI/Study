\section{Выводы}

В ходе выполнения лабораторной работы я исследовал статистические свойства большого текстового корпуса и проверил справедливость закона Ципфа на практике.

\begin{enumerate}
	\item \textbf{Работа с Big Data:} я научился применять методы потоковой обработки данных в Python. Переход от загрузки всего файла в память к использованию генераторов позволил обработать 60 миллионов записей на обычном ПК, избежав переполнения памяти.
	\item \textbf{Лингвистический анализ:} я убедился, что закон Ципфа выполняется для собранного корпуса: график распределения в логарифмической шкале представляет собой почти прямую линию. Наилучшее соответствие наблюдается в области редких слов (отклонение всего 43\%).
	\item \textbf{Интерпретация отклонений:} анализ показал, что наибольшие отклонения (более 300\%) возникают в средней зоне частот, что указывает на специфичность лексики корпуса (IT и политика), используемой чаще, чем в среднем по языку.
	\item \textbf{Практическая значимость:} понимание частотного распределения необходимо для оптимизации поискового индекса; высокочастотные слова (стоп-слова) создают нагрузку на индекс и несут мало информации, поэтому их исключение является важным шагом оптимизации.
\end{enumerate}

\pagebreak
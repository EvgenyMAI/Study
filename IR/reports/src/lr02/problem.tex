\CWHeader{Лабораторная работа \textnumero 2 \enquote{Поисковый робот}}

Необходимо написать парсер на любом языке программирования.
\begin{itemize}
	\item Написать поисковый робот --- компоненты обкачки документов, используя любой язык программирования;
	\item Единственным аргументом поисковому роботу подаётся путь до yaml-конфига, содержащий:
	\begin{itemize}
		\item Данные для базы данные в секции db;
		\item Данные для робота в секции logic: задержка между обкачкой страницы;
		\item Любые другие данные, необходимые для реализовации логики поискового робота.
	\end{itemize}
	\item Сохранять в базе данных (например, MongoDB) документы со следующими полями:
	\begin{itemize}
		\item url, нормализованный;
		\item \enquote{сырой} html-текст документа;
		\item название источника;
		\item Дата обкачки документа в формате Unix time stamp.
	\end{itemize}
	\item Поисковый робот можно остановить в любой момент и при повторном запуске робот должен начать с того документа, с которого он остановился;
	\item Периодически он должен уметь переобкачивать документы, которые уже есть в базе, но только в том случае, если они изменились.
\end{itemize}

\pagebreak
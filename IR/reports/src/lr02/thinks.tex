\section{Выводы}

В ходе выполнения лабораторной работы \textnumero 2 я реализовал полноценный поисковый робот и сформировал базу данных, достаточную для выполнения всех последующих заданий курса.

\begin{enumerate}
	\item \textbf{Навыки разработки:} мною был написан устойчивый к ошибкам crawler на Python. Я освоил работу с NoSQL базой данных MongoDB, включая создание индексов для ускорения поиска по URL и хешу.
	\item \textbf{Оптимизация хранения:} реализован механизм вычисления \texttt{content\_hash} (MD5), который позволяет избегать дублирования данных и обновлять в базе только реально изменившиеся страницы, экономя дисковое пространство и время записи.
	\item \textbf{Работа с Legacy-данными:} важным этапом стала разработка скрипта миграции. Это позволило объединить результаты первой лабораторной работы (статический корпус) с возможностями робота.
\end{enumerate}

\pagebreak
\CWHeader{Лабораторная работа \textnumero 8 \enquote{Булев поиск}}

Нужно реализовать \textbf{позиционный индекс} для полнотекстового поиска с поддержкой фразовых запросов и запросов близости (proximity queries). Индекс должен строиться на основе токенизированного корпуса документов и хранить не только список документов для каждого термина, но и позиции (offset) каждого вхождения термина в документе.

Для построения индекса потребуется выбрать структуры данных для хранения терминов, списков документов и списков позиций. Необходимо описать их в отчёте, указать достоинства и недостатки выбранного метода. Необходимо реализовать все структуры данных самостоятельно, без использования готовых контейнеров STL (\texttt{std::map}, \texttt{std::unordered\_map}, \texttt{std::vector} и т.д.).

Индекс должен поддерживать следующие операции:

\begin{itemize}
	\item \textbf{Фразовый поиск} — точное совпадение последовательности терминов (например, \enquote{машинное обучение}).
	\item \textbf{Proximity запросы} — термины на расстоянии не более N слов друг от друга (например, \enquote{python /5 программирование} означает, что между \enquote{python} и \enquote{программирование} не более 5 слов).
	\item \textbf{Упорядоченный proximity} — термины должны идти в указанном порядке и на заданном расстоянии.
	\item \textbf{Ранжирование} — учет количества и близости вхождений терминов при сортировке результатов.
\end{itemize}

\pagebreak
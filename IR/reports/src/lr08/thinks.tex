\section{Выводы}

В ходе выполнения лабораторной работы я получил практический опыт создания полноценной системы булева поиска с двумя интерфейсами и научился проектировать архитектуру поисковых систем.

\begin{enumerate}
	\item \textbf{Проектирование архитектуры:} я научился разделять систему на независимые компоненты (поисковый движок, парсер запросов, интерфейсы), что обеспечивает гибкость и возможность развития. Использование индекса из ЛР7 показало важность модульности: поисковый движок не зависит от способа построения индекса и работает с любой реализацией BooleanIndex.
	\item \textbf{Реализация двух интерфейсов:} создание CLI и Web-интерфейсов показало разные подходы к взаимодействию с пользователем. CLI удобен для автоматизации, но медленный из-за перезагрузки индекса ($\sim$25--35 сек на запрос), тогда как Web-интерфейс держит индекс в памяти и дает ускорение в \textbf{$\sim$50x} (50--2000 мс), делая систему пригодной для интерактивного использования.
	\item \textbf{Парсинг булевых выражений:} я реализовал лексер и парсер с рекурсивным спуском для разбора запросов с поддержкой приоритетов операторов (NOT $>$ AND $>$ OR) и скобок. Несмотря на то, что в финальной версии для совместимости используется простой парсер, были изучены принципы построения парсеров и важность корректной обработки приоритетов.
	\item \textbf{Оптимизация производительности:} работа с корпусом 41\,818 документов и 2.7M терминов показала критическую важность хранения индекса в памяти. Web-сервер с резидентным индексом устраняет проблему долгой загрузки и обеспечивает приемлемое время ответа.
	\item \textbf{Узкие места и направления улучшений:} выявлены проблемы (медленная NOT операция из-за универсального множества документов, отсутствие ранжирования, лимит вывода результатов), а также предложены улучшения: битовые маски для NOT, TF-IDF/BM25 для ранжирования, пагинация и кэширование.
	\item \textbf{Тестирование и валидация:} разработан набор из 7 автотестов, покрывающих AND/OR/NOT, неявный AND и пустой результат; все тесты пройдены успешно, что подтверждает корректность реализации.
\end{enumerate}

В результате я создал работающую систему булева поиска, способную обрабатывать запросы по корпусу 41\,818 документов за доли секунды (Web-интерфейс). Система успешно прошла все автотесты и готова к использованию для точного булевого поиска.

\pagebreak
\section{Выводы}

В ходе выполнения лабораторной работы я получил глубокий практический опыт работы с морфологическими алгоритмами и оценил их влияние на качество поиска.

\begin{enumerate}
	\item \textbf{Реализация алгоритма Портера:} я научился реализовывать лингвистические алгоритмы с нуля, работая напрямую с UTF-8 на байтовом уровне, и понял важность точной работы с кодировками в международных системах.
	\item \textbf{Морфологический анализ:} я глубоко изучил структуру русского языка (окончания существительных, глаголов, прилагательных, причастий). Реализация 70+ правил удаления суффиксов научила систематическому подходу к обработке языковых конструкций, а концепции RV/R1/R2 регионов дали инструмент для корректного определения границ морфем.
	\item \textbf{Компромиссы качества и производительности:} я увидел trade-off между полнотой и точностью поиска: стемминг улучшил Recall на +42\%, но снизил Precision на -7\%, что показало необходимость балансировки метрик под задачу.
	\item \textbf{Проблемы агрессивного стемминга:} выявлены ограничения алгоритма Портера (не учитывает части речи, создает слишком короткие основы, не обрабатывает беглые гласные), поэтому для production-систем требуется полноценная лемматизация с морфологическим разбором.
	\item \textbf{Оптимизация производительности:} работа с 41\,684 документами показала критическую важность оптимизации; применение техники отложенной дедупликации дало ускорение порядка $\sim$100x по сравнению с наивным подходом.
	\item \textbf{Метрики качества поиска:} освоена оценка поисковых систем через Precision, Recall и F1-score, а также понимание того, что для разных задач приоритет метрик может отличаться.
\end{enumerate}

\pagebreak
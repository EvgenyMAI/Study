\CWHeader{Лабораторная работа \textnumero 7 \enquote{Булев индекс}}

Нужно реализовать \textbf{булев индекс} для полнотекстового поиска с поддержкой логических операций AND, OR и NOT. Индекс должен строиться на основе токенизированного корпуса документов и позволять эффективно выполнять булевы запросы.

Для построения индекса потребуется выбрать структуры данных для хранения терминов и списков документов (posting lists). Необходимо описать их в отчёте, указать достоинства и недостатки выбранного метода. Необходимо реализовать все структуры данных \textbf{самостоятельно}, без использования готовых контейнеров STL (\texttt{std::map}, \texttt{std::unordered\_map}, \texttt{std::set} и т.д.).

Индекс должен поддерживать следующие операции:
\begin{itemize}
	\item \textbf{AND} (пересечение) --- документы, содержащие все указанные термины.
	\item \textbf{OR} (объединение) --- документы, содержащие хотя бы один из терминов.
	\item \textbf{NOT} (отрицание) --- исключение документов, содержащих указанный термин.
\end{itemize}

Привести примеры запросов, которые выполняются неэффективно (слишком долго или потребляют много памяти), объяснить причины и предложить способы оптимизации.

\pagebreak